\documentclass[12pt]{article}
\usepackage[utf8]{inputenc}
\usepackage[T1]{fontenc}
\usepackage[brazil]{babel}
\usepackage{geometry}
\usepackage{hyperref}
\usepackage{enumitem}
\usepackage{listings}
\usepackage{xcolor}
\usepackage{textcomp} 
\usepackage{titlesec}
\usepackage{forest}

\geometry{a4paper, margin=2.5cm}
\hypersetup{
    colorlinks=true,
    linkcolor=blue,
    urlcolor=blue
}
\titleformat{\section}{\large\bfseries}{\thesection}{1em}{}

\lstset{
    escapeinside={/*@}{@*/},  
    basicstyle=\footnotesize\ttfamily,  
    numberstyle=\tiny,      
    frame=single,           
    breaklines=true,           
    postbreak=\mbox{\textcolor{red}{$\hookrightarrow$}\space},
    inputencoding=utf8,        
    literate={├}{{\textcolor{red}{├}}}1 
}

\title{\textbf{VINAc – Arquivador com Suporte à Compressão}}
\author{Pietro Comin \\ GRR20241955 \\ \texttt{pietro.comin@ufpr.br}}
\date{}

\begin{document}

\maketitle

\noindent \textbf{Número de linhas do código:} 1989

\section*{Descrição}

O \textbf{VINAc} é um arquivador com suporte à compressão baseado em linha de comando, semelhante a ferramentas como \texttt{tar}, \texttt{zip} ou \texttt{rar}. Seu objetivo é permitir o arquivamento de múltiplos arquivos em um único container (\texttt{.vc}), com a opção de compressão individual utilizando o algoritmo LZ.

A estrutura do arquivo \texttt{.vc} inclui uma área de diretório responsável por armazenar todos os metadados necessários para manipulação dos membros arquivados.

\section*{Funcionalidades}

O programa \texttt{vinac} oferece suporte às seguites opções de execução:

\begin{itemize}[noitemsep]
    \item \texttt{-ip} ou \texttt{-p}: insere membros sem compressão
    \item \texttt{-ic} ou \texttt{-i}: insere membros com compressão (LZ)
    \item \texttt{-m membro}: move um membro dentro do arquivo
    \item \texttt{-x}: extrai membros (ou todos)
    \item \texttt{-r}: remove membros do arquivo
    \item \texttt{-c}: lista o conteúdo do arquivo
\end{itemize}

\section*{Estrutura do Projeto}

\begin{forest}
  for tree={
    font=\ttfamily,
    grow'=0,
    child anchor=west,
    parent anchor=south,
    anchor=west,
    calign=first,
    edge path={
      \noexpand\path [draw, \forestoption{edge}]
      (!u.south west) +(7.5pt,0) |- (.child anchor) \forestoption{edge label};
    },
    before typesetting nodes={
      if n=1
        {insert before={[,phantom]}}
        {}
    },
    fit=band,
    before computing xy={l=15pt},
  }
[A1/
  [include/
    [aux.h]
    [lz/
      [lz.h]
    ]
    [options.h]
    [types.h]
    [utils.h]
    [vina.h]
  ]
  [src/
    [aux.c]
    [main.c]
    [options.c]
    [utils.c]
    [vina.c]
    [lz/
      [lz.c]
    ]
  ]
  [build/
    [objects/ (gerado durante a compilação)]
  ]
  [login/
    [vinac (executável gerado)]
  ]
  [makefile]
  [A1 - O Arquivador VINAc.pdf (enunciado do trabalho)]
  [texto.txt]
  [texto2.txt]
  [texto3.txt]
  [texto4.txt]
]
\end{forest}

\section*{Compilação}

Para compilar o projeto, utilize:

\begin{lstlisting}[language=bash]
make
\end{lstlisting}

Para limpar os arquivos gerados:

\begin{lstlisting}[language=bash]
make clean
\end{lstlisting}

\section*{Estruturas de Dados e Algoritmos}

\subsection*{Estruturas de Dados}

\begin{itemize}
    \item \texttt{struct arquivo}: armazena os metadados de cada membro do arquivo:
    \begin{itemize}
        \item nome
        \item UID
        \item tamanho original
        \item tamanho comprimido
        \item ordem
        \item offset no arquivo
        \item data de modificação
    \end{itemize}
    \item \texttt{struct diretorio}: vetor de ponteiros para \texttt{arquivo}, gerencia todos os membros do container.
\end{itemize}

\subsection*{Funções Relevantes}

A manipulação de arquivos variáveis em tamanho e posição foi um dos principais desafios. As seguintes funções foram criadas para garantir consistência:

\begin{itemize}
    \item \texttt{int move(...)}: move um bloco dentro do arquivo
    \item \texttt{int move\_sequencial(...)}: move membros em sequência, evitando sobrescrita
    \item \texttt{int insere\_membro\_arq(...)}: insere um novo membro
    \item \texttt{int comprime\_arquivo(...)}: comprime com LZ e atualiza metadados
    \item \texttt{int descomprime\_arquivo(...)}: descomprime e restaura o conteúdo
\end{itemize}

\section*{Decisões de Implementação}

\begin{itemize}
    \item A compressão LZ é usada apenas se houver ganho no tamanho.
    \item A área de diretório é mantida em RAM para facilitar o controle.
    \item Os dados dos membros são manipulados diretamente no disco.
    \item O código é modularizado para facilitar testes e manutenção.
\end{itemize}

\section*{Dificuldades Encontradas}

\begin{itemize}
    \item \textbf{Deslocamentos binários}: o gerenciamento de offsets exigiu lógica cuidadosa e criação de funções específicas, já que no máximo um membro pode ser deslocado por vez, evitando uso de buffers muito grandes.
    \item \textbf{Compressão seletiva}: inserir membros apenas quando a compressão for vantajosa envolveu lógica condicional.
    \item \textbf{Atualização de metadados}: todas as operações exigem atualização precisa de offsets e ordem.
    \item \textbf{Atomicidade}: foi necessário evitar o uso de arquivos temporários, otimizando a leitura e escrita direta em disco.
\end{itemize}

\section*{Bugs Conhecidos}

\begin{itemize}
    \item \textbf{Warning no Valgrind}: alertas sobre bytes não inicializados. Após várias sessões de depuração, acredita-se que seja inofensivo.
    \item \textbf{Testes limitados}: ainda não há testes com arquivos muito grandes ou com nomes inválidos.
\end{itemize}

\section*{Contato}

Para dúvidas ou sugestões, entre em contato: \href{mailto:pietro.comin@ufpr.br}{pietro.comin@ufpr.br}

\end{document}